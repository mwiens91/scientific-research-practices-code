% Set up the document
\documentclass{article}

% Page size
\usepackage[
    letterpaper,]{geometry}

% Lines between paragraphs
\setlength{\parskip}{\baselineskip}
\setlength{\parindent}{0pt}

% Math
\usepackage{mathtools}
\usepackage{amssymb}
\usepackage{commath}

% Math shortcuts
\newcommand{\tT}{\text{T}}

% Links
\usepackage{hyperref}

\begin{document}

We will be primarily interested in determining $\hat{p}_{\tT,s}$, which
we define as the \textit{steady state} proportion of hypotheses with
tally $s$ that are true.

Let $A$ be the number of agents in the model, and let
$n_t, n_{t, \tT, s}$ denote the total number of hypotheses and the
number of true hypotheses with tally $s$, respectively, during the
$t$-th iteration of the model. To simplify notation we fix a time step
$t$ and suppress the $t$ subscripts by defining, for any variable $x_t$,
%
\begin{align*}
    x &\coloneqq x_t \\
    x^\prime &\coloneqq x_{t + 1}
\end{align*}
%
Define $$f_{\tT, s} \coloneqq \frac{n_{\tT, s}}{n}$$ $$a \coloneqq \frac{A}{n}$$
where $f_{\tT, s}$ is the frequency of true hypotheses with tally $s$
and $a$ is the number of agents in proportion to the number of
hypotheses.

Fixing any $s$, we have the following contributions to the total number
of true hypotheses with this tally after each time step:

\begin{itemize}
    \item hypotheses with tally $s - 1$ that were successfully
        replicated:
        $$a n r f_{\tT, s - 1} \del{1 - \beta} c_{R +}$$
    \item hypotheses with tally $s + 1$ that failed to replicated:
        $$a n r f_{\tT, s + 1} \beta c_{R -}$$
    \item if $s = 1$, new hypothesis with positive results that were
        published:
        $$a n \del{1 - r} b \del{1 - \beta}$$
    \item if $s = -1$, new hypothesis with negative results that were
        published:
        $$a n \del{1 - r} b \beta c_{N -}$$
\end{itemize}

However, we also need to subtract the number of hypotheses with tally
$s$ that underwent a replication attempt:
$$a n r f_{\tT, s} \del{\del{1 - \beta} c_{R +} + \beta c_{R -}}$$

Note that, to simplify the analysis, we are assuming that during any
given time step there is \textit{at most} one replication attempt per
tested hypothesis. While there are no such restrictions in the model,
this approach is valid in the limit that there are far more tested
hypotheses than agents, which is precisely what we are considering when
analysing the model's long term behaviour.

Moving forward, we have the following recurrence relation:

For $s = 1$:
%
\begin{align*}
    n^\prime_{\tT, 1}
        = &n_{\tT, 1} \\
          &+ a n r \del{
                - f_{\tT, 1} \del{\del{1 - \beta} c_{R +} + \beta c_{R -}}
                + f_{\tT, 0} \del{1 - \beta} c_{R +}
                + f_{\tT, 2} \beta c_{R -}
          } \\
          &+ a n \del{1 - r} b \beta c_{N -}
\end{align*}
%
For $s = - 1$:
%
\begin{align*}
    n^\prime_{\tT, 1}
        = &n_{\tT, -1} \\
          &+ a n r \del{
            - f_{\tT, -1} \del{\del{1 - \beta} c_{R +} + \beta c_{R -}}
            + f_{\tT, -2} \del{1 - \beta} c_{R +}
            + f_{\tT, 0} \beta c_{R -}
          } \\
          &+ a n \del{1 - r} b \del{1 - \beta}
\end{align*}
%
For $s \notin \cbr{1, -1}$:
%
\begin{align*}
    n^\prime_{\tT, 1}
        = &n_{\tT, s} \\
          &+ a n r \del{
              - f_{\tT, s} \del{\del{1 - \beta} c_{R +} + \beta c_{R -}}
              + f_{\tT, s - 1} \del{1 - \beta} c_{R +}
              + f_{\tT, s + 1} \beta c_{R -}
          }
\end{align*}

\end{document}
